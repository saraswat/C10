C10 syntax and informal semantic
Vijay Saraswat
Last update Wed Sep 17 05:40:25 EDT 2014




From a syntax point of view, the language is like that of Prolog with
the following major changes:

  (a) Object-orientation -- a program is made up of a number of class
      and interface declarations, organized in packages. Each class
      defines (static or instance) fields,  methods and
      constructors. Method definitions may be abstract, overloaded and
      overriding. An interface defines abstract methods; classes
      implement interfaces. Classes realize a single-inheritance
      hierachy but may implement multiple interfaces.   

      Methods and fields may be static or instance.

      TODO: Determine if traits are worth introducing.

  (b) Strong typing: All expresions have a compile-time type. The
      compiler checks that only operations permitted by the type are
      performed on an expression. 

  (c) A distinction is made between the syntactic categories of
      agents, goals and constraints. (Prolog permits only goals.) This 
      reflects the basis of C10 in a richer subset of logic than
      definite clauses. Users may define symbols in all these categories. 

A compilation unit is either a class definition, a struct definition,
an interface definiton or a type definition.

Classes, structs and interfaces are container types -- their bodies
contain constructors, and static or instance type, field, method and
function definitions.





Predicate and function methods are declared thus. 










CONSTRUCTOR definitions






Additional Static Semantics Rules
(over what X10 already provides)


OO structure
The user defines units -- classes, structs and interfaces -- as in X10. The bodies 
of units are as in X10. TBD: Consider adding traits rather than interfaces. 


EXAMPLE CODE

The class Set is defined with various operators and static methods.

public class Set[T] {
  public static type a[T](s:Set[T])=T{self in s}.

  public static def singleton[T](t:T)                    =collect offer t.
  public static native def collect[T](a:Offers[T]):Set[T].

  public operator this * (o:Set[T])=collect all(b:a[T](this)) if (some(c:a[T](y))b=c) offer b.
  public operator this + (o:Set[T])=collect {all(x:a[T](this)) offer x, all(x:a[T](o)) offer y}.
  public operator this - (o:Set[T])=collect all(x:a[T](this)) if (all(y:a[T](o))x !=y) offer x.
  public def select[S](f:(T,S)=>S, zero:T):S = this as List[T].select(f,zero).

}

public class List[T] {
   public def select[S](f:(T,S)=>S, zero:T):S =
      this.null ? zero ^ tail.select(f, f(head, zero)).
}

Below we assume the follownig static type definitions:

   static type Int(x:Int,y:Int{x<=y})   = Int{self >=x, self<=y}.
   static type Long(x:Long,y:Long{x<=y})= Long{self >=x, self<=y}.
   static type (x:Int)..(y:Int{x<=y})   = Int(X,Y).
   static type (x:Long)..(y:Long{x<=y}) = Long(X,Y).

(Note that x..y is now ambiguously a type or an expression.)
   
Map /GroupBy / Reduce computations.



This definition of prime numbers in the range 1..N checks for each candidate x whether
all the numbers from 2 to x/2 are factors. 

def prime(N:Int{self >=2}) = 
   collect all (x:2..N)
            if (all (y:2..x/2) y !% x)
              offer x.


This definition uses the Sieve of Eratosthenes to push the checking into the generator,
and only generate those numbers that are prime (i.e. do not have prime factors smaller
than themselves as factors). This shows the advantage of decoupling 
the offer statement (the generator) from the set collector, and the conditions
(these are captured via agent/goal interplay in all the sophisticated ways the 
language permits).

def prime(N:Int) = collect prime(2..N).
def prime(P:List[Int]) {
  if (P.size>1) {
    offer P.head, 
    prime(P.tail.filter((x:Int)=>x%P.head >0))
  }
}

Below we assume the following definitions.

TODO: Check if this works with the X10 parser.
We assume that a ternary operatorn _ ? _^_ has been defined thus
  static operator [T] (a:Goal)?(b:T)^(c:T):T = M where if (a) M=b else M=c.

Now various aggregators such as max and min can be defined thus. In the actual
implementation this computation will be pushed into the generator.

   def max[T](x:Agent[Offers[T]]){T <: Arithmetic[T]} = 
      collect(x).select((x,y)=>(x<y)?y^x, T.maxValue).

   def argmin[S,T](f:(S)=>T){T <: Arithmetic[T], nullable S} = 
      collect(all (x:S) offer Pair(f(x),x))
         .select((u,v)=>(u.fst<v.fst)?u^v, Pair(T.maxValue, S.null))
         .snd.

class KMeans(N:Int, P:Int, K:Int, pts:Rail[Vector(N)](P)) {
  type Vector = Rail[Double](N). 

  agent delta(A:Vector, B:Vector)= max(all(i:A.domain) offer Math.abs(A(i)-B(i))).

  agent kmeans=kmeans(collect all(i:0..(K-1)) offer pts(i) groupby i).

  agent kmeans(old:Rail[Vector](K))= {
    T = avg(all(i:0..(P-1))
           offer pts(i) groupby 
             argmin ((j:0..(K-1))=> delta(pts(i), old(j)))),
     X = delta(old, T),
     offer X < epsilon? T ^ kmeans(T)
   }
}






